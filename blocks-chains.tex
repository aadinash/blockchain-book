\chapter{Blocks and Chains}

{\color{red}
\begin{itemize}
\item The decentralized setting; the network
\item Adversarial and honest nodes, corruption, sybil attacks
\item Network delays, message delivery
\item Broadcasting transactions to everyone - be your own bank
\item Money arising out of social consensus
\item The double spending attack
\item Simple ideas don't work: Serializing transactions naively, the critical time Δ
\item Proof-of-work as a consensus mechanism without blocks
\item The honest computational majority assumption
\item The proof-of-work equation; the parameters T and κ
\item The block
\item The chain
\item The longest chain rule
\item The mempool
\item Temporary forks / reorgs
\item Coming to an agreement: Blockchain convergence
\item The genesis block; the arrow of time
\item Verifying chains
\item The Chain Growth, Common Prefix, and Chain Quality, intuitively
\item Transaction confirmation rules; the Common Prefix parameter k
\item The minority double spending attack and its inevitable failure; negligibility in k
\item Maintaining the UTXO under reorgs
\end{itemize}
}
