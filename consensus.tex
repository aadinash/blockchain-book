\chapter{Blockchains are Secure}

{\color{red}
\begin{itemize}
\item The backbone model in the static difficulty setting
\item Interactive Turing machines
\item The environment
\item Synchronous time: The integer round
\item The theoretical network model
\item The rushing adversary
\item The Sybil adversary
\item The parties: The parameters $n$, $t$
\item The honest majority assumption: The parameter $\delta$
\item Mining modeled as a Random Oracle: The parameter $q$
\item The backbone protocol
\item Validating blocks
\item The longest chain rule
\item Mining
\item The Chain Growth property: The parameters $s$ and $\tau$
\item The Common Prefix property: The parameter $k$
\item The Chain Quality property: The parameters $\ell$ and $\mu$
\item Ledger liveness: The parameter $u$
\item Ledger safety
\item Proof of liveness from Chain Growth and Chain Quality
\item Calculation of the liveness parameter $u$
\item Proof of safety from Common Prefix
\item Successful rounds and uniquely successful rounds
\item The probabilistic treatment using the random variables $X$, $Y$, and $Z$
\item Probabilities of success and failure
\item Chernoff bound intuition
\item Chernoff bound theorem for binomial distributions: The parameter $\epsilon$
\item Convergence opportunities
\item Uniquely successful rounds avoid fan-out attacks unless matched by adversarial mining power
\item The equal computational split model
\item The world is a good place: Typical executions
\item The distance between $X$ and $Y$: The block production parameter $f$
\item The distance between $Y$ and $Z$: The Chernoff error parameter $\epsilon$
\item The Chernoff waiting time $\lambda$
\item The balancing equation $3\epsilon + 3f < \delta$
\item Calculating the relationship between adversarial power $(n, t)$, network diameter, and block production rate
\item The Typicality Theorem
\item The Chain Growth Lemma
\item A proof of the Chain Growth property, calculation of the parameters $s=\lambda$ and $\tau=(1 - \epsilon)f$
\item The Chain Slowness Lemma
\item The Pairing Lemma
\item A proof of the Common Prefix property
\end{itemize}
}
